\usepackage[a4paper, left=1cm, right=1cm, top=3cm, marginparwidth=2cm]{geometry}
\usepackage{marginnote}

\setcounter{secnumdepth}{3} % default value for 'report' class is "2"


\usepackage[utf8]{inputenc}
\usepackage[T1]{fontenc}
\usepackage{tikzsymbols}
\usepackage[backend=bibtex, style=numeric, sorting=none]{biblatex} 
\usepackage{csquotes}
\setlength\bibitemsep{\baselineskip}
\addbibresource{References.bib}

\usepackage{float} % to take away the error unknown float option `H'
% https://stackoverflow.com/a/27243065/505306
% Making caption font smaller on figures and tables.
\usepackage{caption}
\captionsetup{font=footnotesize}

%-------------------------------------------------------------------------------
%\usepackage[left,pagewise]{lineno}
%\usepackage[right]{lineno}
%\linenumbers
%--------------------------------------------------------------------------------
\usepackage[framemethod=TikZ]{mdframed}
\usepackage{wrapfig}
\usepackage{amsthm}
\usepackage{amsfonts}
% ------------------------------------------------------------------------------
\usepackage[toc,page]{appendix}
\usepackage[stretch=10]{microtype} % https://tex.stackexchange.com/a/586  The great microtype package
%-------------------------------------------------------------------------------
%https://tex.stackexchange.com/a/278199/17858
% For aligning itemize environments to left
\usepackage{enumitem}
\mdfsetup{%
   middlelinecolor=red,
   middlelinewidth=0pt,
   roundcorner=10pt}

\mdfdefinestyle{MyFrame}{%
    linecolor=black,
    outerlinewidth=0.1pt,
    roundcorner=0pt,
    innertopmargin=14pt,
    innerbottommargin=4pt,
    innerrightmargin=4pt,
    innerleftmargin=4pt,
        leftmargin = 4pt,
        rightmargin = 4pt,
    %backgroundcolor=gray!50!white}
     }
%-------------------------------------------------------------------------------

\definecolor{theocol}{RGB}{255, 222, 228}
\newtheorem{theo}{Theorem}
\newenvironment{ftheo}
  {\begin{mdframed}[style=MyFrame,nobreak=true, backgroundcolor=theocol]\begin{theo}}
  {\end{theo}\end{mdframed}}


\definecolor{lemcol}{RGB}{226, 255, 220}
\newtheorem{lem}{Lemma}
\newenvironment{flem}
  {\begin{mdframed}[style=MyFrame,nobreak=true, backgroundcolor=lemcol]\begin{lem}}
  {\end{lem}\end{mdframed}}

\definecolor{corcol}{RGB}{227, 227, 227}
\newtheorem{cor}{Corollary}
\newenvironment{fcor}
  {\begin{mdframed}[style=MyFrame,nobreak=true, backgroundcolor=corcol]\begin{cor}}
  {\end{cor}\end{mdframed}}

\definecolor{propcol}{RGB}{227, 227, 227}
\newtheorem{prop}{Proposition}
\newenvironment{fprop}
  {\begin{mdframed}[style=MyFrame,nobreak=true, backgroundcolor=propcol]\begin{prop}}
  {\end{prop}\end{mdframed}}

\definecolor{conjcol}{RGB}{248, 218, 194}
\newtheorem{conj}{Conjecture}
\newenvironment{fconj}
  {\begin{mdframed}[style=MyFrame,nobreak=true, backgroundcolor=conjcol]\begin{conj}}
  {\end{conj}\end{mdframed}}


\definecolor{notecol}{RGB}{255, 152, 149  }
\newenvironment{note}
  {\begin{mdframed}[style=MyFrame,nobreak=true, backgroundcolor=notecol]}
  {\end{mdframed}}

\usepackage{enumitem}

%-------------------------------------------------------------------------------
% https://tex.stackexchange.com/q/2291/17858
% if you want to create a new list from scratch
\newlist{alphalist}{enumerate}{1}
% in that case, at least label must be specified using \setlist
\setlist[alphalist,1]{label=\textbf{\Alph*.}}

%--------------------------------------------------------------------------------
\usepackage[english]{babel}
%\setlength{\voffset}{-0.75in}
\setlength{\headsep}{5pt}

%-------------------------------------------------------------------------------
\usepackage{blindtext}
\usepackage{lipsum}
\usepackage{kantlipsum}
\usepackage{textcomp}

\usepackage{amssymb}
\usepackage{amsmath}
\usepackage{IEEEtrantools}
\usepackage{mathtools}

\usepackage{listings}


%-----------------------------------------------------------------------------------------------
\usepackage[inline]{asymptote}
\usepackage{asypictureB}
\usepackage{filecontents}
\usepackage{parskip} 
\usepackage{tocloft}
\usepackage{graphicx}                             % Include images into pdf document 
\usepackage{tikz}                                 % Useful for the circled command. Don't remove!
\usepackage{multicol}                             % For lists in two or more columns


%% top right numbering of pages https://tex.stackexchange.com/a/56321
\pagestyle{myheadings}



\usepackage[nokwfunc, ruled, linesnumbered,nofillcomment]{algorithm2e}
\newcommand\mycommfont[1]{\footnotesize\ttfamily\textcolor{gray}{#1}}
\SetCommentSty{mycommfont}

\usepackage[T1]{fontenc}
%------------------------------------------------------------------------------------------------
\usepackage{needspace}                % So that sections/code-blocks don't straddle two pages 
\usepackage{mathtools}
\usepackage{subfig}
\usepackage{etoolbox}
\usepackage{color}
\usepackage{pifont}
\setlength{\parindent}{20pt}
% For italicizing quotes: https://tex.stackexchange.com/a/288556/17858

%------------------------------------------------------------------------------------------------
\usepackage{quoting,xparse}  %%% from https://tex.stackexchange.com/a/391739

\NewDocumentCommand{\bywhom}{m}{% the Bourbaki trick
  {\nobreak\hfill\penalty50\hskip1em\null\nobreak
   \hfill\mbox{\normalfont(#1)}%
   \parfillskip=0pt \finalhyphendemerits=0 \par}%
}

\NewDocumentEnvironment{pquotation}{m}
  {\begin{quoting}[
     indentfirst=true,
     leftmargin=\parindent,
     rightmargin=\parindent]\itshape}
  {\bywhom{#1}\end{quoting}}


%------------------------------------------------------------------------------------------------
% Convenience commands
\newcommand*\circled[1]{\tikz[baseline=(char.base)]{\node[shape=circle,draw,inner sep=2pt] (char) {#1};}}
\providecommand{\myceil}[1]{\left \lceil #1 \right \rceil }	% Ceil function
\providecommand{\myfloor}[1]{\left \lfloor #1 \right \rfloor }	% Floor function\renewcommand{\labelitemi}{\tiny$\blacksquare$}	
\newcommand\given[1][]{\:#1\vert\:}                             % for drawing the conditional probability `|` sign neatly.
\newcommand\RR{\mathbb{R}}					% Set of reals numbers
\newcommand\CC{\mathbb{C}}					% Set of complex numbers
\newcommand\ZZ{\mathbb{Z}}					% Set of integers
\newcommand\NN{\mathbb{N}}					% Set of naturals
\newcommand\rarr{\rightarrow}					% Rightarrow
\newcommand\larr{\leftarrow}					% Leftarrow
\newcommand\defeq{\coloneqq}					% := symbol
\renewcommand\tilde{ \: \thicksim \: }				% Sane tildas

\newmdenv[topline=false, bottomline=false, skipabove=\topsep,skipbelow=\topsep]{siderules}

% https://tex.stackexchange.com/a/458876/17858 rounded
% pink rectangles around an inline word
\newcommand{\sticker}[1]{\tikz[baseline=(X.base)]\node [draw=red,fill=pink!60,semithick,
     rectangle,inner sep=2pt, rounded corners=3pt] (X) { {\footnotesize \color{red} #1}};}

\newif\ifshowcode
\showcodetrue
\usepackage{latexsym}
\usepackage{listings}
\usepackage{color}
\definecolor{linkcolor}{rgb}{0.7, 0, 0}

\usepackage{todonotes}
\usepackage{booktabs}

\usepackage[%
raiselinks,%
pdfhighlight=/O,%
hyperfigures,%
breaklinks,%
colorlinks,%
pdfstartview=FitBH,%
linkcolor={linkcolor},%
anchorcolor={linkcolor},%
citecolor={linkcolor},%
filecolor={linkcolor},%
menucolor={linkcolor},%
urlcolor={linkcolor}%
]{hyperref}

% taken from https://tex.stackexchange.com/a/371469 
% fior drawing a nice rule across the page
\newcommand\myrule{\par\noindent\rule{\textwidth}{0.4pt}}

%---------------------------------------------------------
\usepackage{xcolor}
\usepackage{sectsty}
\allsectionsfont{\sffamily}
\definecolor{lava}{rgb}{0.81, 0.06, 0.13}
\definecolor{mahogany}{rgb}{0.75, 0.25, 0.0}
\definecolor{sacramentostategreen}{rgb}{0.0, 0.34, 0.25}
%---------------------------------------------------------


%-------------------------------------
%\usepackage[toc]{multitoc}
%\renewcommand*{\multicolumntoc}{2}
%\setlength{\columnsep}{1cm}
%\setlength{\columnseprule}{0.1pt} % for a vertical separator between the columns
%------------------------------------

\setcounter{tocdepth}{2}

%-----------------------------------------------------------------------------------------
% FOR SOURCE CODE FORMATTING
\definecolor{cosmiclatte}{rgb}{1.0, 0.97, 0.91}
\definecolor{asparagus}{rgb}{0.53, 0.66, 0.42}
\lstdefinestyle{numbers} {numbers=left, stepnumber=1, numberstyle=\tiny, numbersep=10pt}
\lstdefinestyle{MyFrame}{backgroundcolor=\color{cosmiclatte},frame=shadowbox}

\lstdefinestyle{MyPythonStyle} {language=Python,style=MyFrame,frame=lines}
\lstset{language=Python,
                basicstyle=\ttfamily,
                keywordstyle=\color{blue}\ttfamily,
                stringstyle=\color{red}\ttfamily,
                commentstyle=\color{asparagus}\ttfamily,
                morecomment=[l][\color{magenta}]{\#}}


\lstdefinestyle{MyHaskellStyle} {language=Haskell,style=MyFrame,frame=lines}
\lstset{language=Haskell,
                basicstyle=\ttfamily,
                keywordstyle=\color{blue}\ttfamily,
                stringstyle=\color{red}\ttfamily,
                commentstyle=\color{asparagus}\ttfamily,
                morecomment=[l][\color{magenta}]{\#}}
%-----------------------------------------------------------------------------------------
\usepackage{marginnote}

\newcommand\margin[1]{\marginnote{\color{cadmiumgreen}{#1} }}				% Sane tildas
\definecolor{cadmiumgreen}{rgb}{0.0, 0.42, 0.24}
%---------------------------------------------------------------------------------------------------
\renewcommand{\labelitemi}{\scriptsize$\blacksquare$} % squaremarkers for bullets of itemized lists


%-------------------------------------------------------------------------------
% To express an idea in a crunchy way.
\newcommand{\crunchy}[1]{\lbrack{} \large \textit{#1} \normalsize \rbrack} 



% For italicizing quotes https://tex.stackexchange.com/a/288556
\usepackage{csquotes}
\renewcommand{\mkbegdispquote}[2]{\itshape}

% for italcizing figure captions
\usepackage[font={small,it}]{caption} % from https://tex.stackexchange.com/a/832


\SetKwComment{Comment}{$\triangleright$\ }{}
\newcommand{\myblue}[1]{{\color{blue}{#1}}}