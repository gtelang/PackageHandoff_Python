\newcommand{\NWtarget}[2]{\hypertarget{#1}{#2}}
\newcommand{\NWlink}[2]{\hyperlink{#1}{#2}}
\newcommand{\NWtxtMacroDefBy}{Fragment defined by}
\newcommand{\NWtxtMacroRefIn}{Fragment referenced in}
\newcommand{\NWtxtMacroNoRef}{Fragment never referenced}
\newcommand{\NWtxtDefBy}{Defined by}
\newcommand{\NWtxtRefIn}{Referenced in}
\newcommand{\NWtxtNoRef}{Not referenced}
\newcommand{\NWtxtFileDefBy}{File defined by}
\newcommand{\NWtxtIdentsUsed}{Uses:}
\newcommand{\NWtxtIdentsNotUsed}{Never used}
\newcommand{\NWtxtIdentsDefed}{Defines:}
\newcommand{\NWsep}{${\diamond}$}
\newcommand{\NWnotglobal}{(not defined globally)}
\newcommand{\NWuseHyperlinks}{}
\documentclass[12.0pt]{report}
\input{standard_settings.tex}

%%% Super useful for marking todo notes, ripped from here: 
%%% https://tex.stackexchange.com/a/178806/17858
\usepackage{xargs}                      % Use more than one optional parameter in a new commands
\usepackage[colorinlistoftodos,prependcaption,textsize=tiny]{todonotes}
\newcommandx{\UNSURE}[2][1=]{\todo[linecolor=blue,backgroundcolor=blue!25,bordercolor=blue,#1]{#2}}
\newcommandx{\TODO}[2][1=]{\todo[linecolor=red,backgroundcolor=red!25,bordercolor=red,#1]{#2}}

\usepackage{kantlipsum}
\usepackage{fancyvrb}
\usepackage{setspace}
\newenvironment{CVerbatim}
 {\singlespacing\center\BVerbatim}
 {\endBVerbatim\endcenter}

\usepackage{tocloft}
\renewcommand{\cftpartfont}{\LARGE\itshape} % Part title in Huge Italic font
\usepackage{hyperref}
\usepackage{etoolbox}
% Better formatting of backticks in 
% verbatim environment. 
\usepackage{upquote}

% page numbering at top right
\usepackage{fancyhdr}
\pagestyle{fancy}
\fancyhf{}
\fancyhead[R]{\thepage}

\begin{document}
\begin{titlepage}
	\centering
        {\Huge Analyses of Experimental Heuristics for Package-Handoff Type Problems\\}
        \vspace{20mm}
        {\Large Gaurish Telang}
\end{titlepage}
\pagenumbering{arabic}
\setcounter{page}{2} 
\setcounter{tocdepth}{1}
\tableofcontents
\addtocontents{toc}{~\hfill\textbf{Page}\par}

\chapter{Overview}

How do you get a package from point $A$ to point $B$ with a fleet of drones each 
capable of various maximum speeds? This is the question we try to answer in all 
its various avatars by developing algorithms, heuristics, local optimality heuristics 
and lower-bounds. 

To be more specific,  we are given as input the positions $P_i$ of $n$ drones in 
the plane each capable of a maximum speed of $u_i$. Also given is a package present 
at $S$ that needs to get to $T$. Each drone is capable of picking up the package and 
flying with speed $u_i$ to some other point to hand it off to another drone. 

\begin{figure}[H]   
  \centering 
  \asyinclude{./asy/dummy-example.asy}
  \caption{An example of the Package Handoff Problem}
\end{figure}    

The challenge is to figure how to get the drones to cooperate to send 
the package from $S$ to $T$ in the least possible time. 

To solve the problem we need to be able to do several things

\begin{itemize}
\item Figure out which subset of the drones are used in the optimal schedule. 
\item Find the order in which the handoffs happend between the drones used in a schedule. 
\item Find the ``handoff'' points when drone $k$ rendezvous with drone $k+1$ 
      to give it the package.  
\end{itemize}

This category of problems is a generalization of computing shortest paths in $\mathbb{R}^2$ 
between 
two points. As far as we know such problems have not been considered before in the 
operations research literature; it is, however, reminescent of the Weighted Region 
Problem (\texttt{WRP}) where one needs to figure out how to compute a shortest 
\textit{weighted} path between two points in the plane
that has been partitioned into convex polygonal regions, each associated with a constant 
multiplicative weight for scaling the euclidean distance between two points 
\textit{within} that region.  


The distinctive feature of this problem and its generalizations is figuring out how 
to make multiple agents of \textit{varying} capabilities  located at different points 
in $\mathbb{R}^2$ (such as maximum capable speed, battery capacity, tethering constraints 
etc.) \textit{cooperate} in transporting one or more packages most efficiently 
from their given sources to their target destinations. 

Each chapter in this document is devoted to developing algorithms for a specific 
variant of the package handoff problem (henceforth abbreviated as \texttt{PHO}), beginning 
with the plain-vanilla single package handoff problem described above. 
For most algorithms we will also be giving implementations in Python described in a 
literate style. 

You can check out the code from the following online GitHub repository: 
\url{https://github.com/gtelang/PackageHandoff_Python}

%-----------------------------------------------------
\chapter{Single Package Handoff}
\label{chap:single-package-handoff}






\section{Chapter Index of Fragments}
None.

\section{Chapter Index of Identifiers}
 
%------------------------------------------------------

\begin{appendices}
\end{appendices}

\end{document}
